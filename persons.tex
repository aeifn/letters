\section{Деев-Хомяковский, Григорий Дмитриевич}
(род. 1888) — литературный деятель, поэт. Член ВКП(б). До Октябрьской революции — пастух, рабочий, потом — педагог, секретарь "Будильника", руководитель "Суриковского литературно-музыкального кружка". После Октября — редактор нескольких журналов и активный член и учредитель разнообразных литературных организаций. Автор ряда исторических и литературно-критических статей, пьес, рассказов, стихотворных сборников. Литературные произведения Д.-Х. по б. ч. лишены художественности. Известность он приобрел в качестве руководителя крестьян-литераторов, узко ориентировавшегося на т. н. самородков-самоучек.
Лит. см. Владиславлев И., Литература великого десятилетия, том I, М.—Л., 1928. 
http://dic.academic.ru/dic.nsf/enc-biography/18962/%D0%94%D0%B5%D0%B5%D0%B2

